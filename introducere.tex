\chapter*{Introducere}


În această lucrare vom încerca să setăm un obiectiv în procesul dezvoltării unor noi tehnologii, prin conștientizarea elementelor care ne definesc ca oameni, în scopul eficienței și atenuării curbei de învățare a interacțiunii cu produsul final.

Corpul uman are la dispoziție un ansamblu de simțuri care îi permit o experiență bogată în interacțiunea cu mediul înconjurător. Ridicând spre exemplu un pahar cu apă, putem acumula o serie de informații cu un minim de efort: vedem cantitatea de apa, îi putem simți greutatea, temperatura, textura etc. Cu toate acestea interacțiunea noastră cu tehnologia are niște constrângeri considerabile, cu telefonul mobil interacționăm în mare parte atingând sticla ecranului cu un singur deget. Pentru a putea folosi un computer trebuie să învățăm să folosim tastatura și mouse-ul, cu alte cuvinte să ne modelăm abilitățile în concordanță cu uneltele pe care le-am creat.

O soluție pentru a scăpa de aceste constrângeri și a eficientiza interacțiunea cu uneltele pe care le creăm este sa le modelăm astfel încât să ne augmenteze capabilitățile umane deja e existente, făcând interacțiunea cât mai naturală posibil.

În prezentarea lui Douglas Engelbart din Decembrie 1968, supranumită și \textit{Mama tuturor demonstrațiilor}, au fost introduse elementele fundamentale ale computerelor personale (mouse-ul, interfața grafică, ferestrele, hipertext-ul, video-conferința, procesarea text, controlul versiunilor), prin care s-a încercat crearea unei interacțiuni cât mai naturale cu computerul. Aceste metode au fost îmbunătățite de-a lungul timpului dar în principiu au rămase aceleași până astăzi.

O idee importantă pe care încercăm să o atingem este foarte bine descrisă de John Culkin, SJ, profesor de comunicare la Universitatea Fordham din New York, în citatul: \begin{quotation}
„Devenim ceea ce admirăm. Ne sculptăm unelte, iar mai apoi uneltele ne sculptează pe noi.”\footnote{Original \textit{„We become what we behold. We shape our tools and then our tools shape us”} - John Culkin.}
\end{quotation}

Acest citat poate fi privit din mai multe unghiuri. La scară largă o tehnologie acceptată și adoptată de societate are puterea de-a schimba mediul în care trăim.

Un exemplu ar fi introducerea automobilelor în viața cotidiană. Lumea s-a micșorat, majoritatea zonelor locuite fiind conectate printr-o rețea amplă de străzi și drumuri rutiere, iar călătoríi ce până atunci ar fi durat zile sau săptămâni, pot fi făcute în doar câteva ore.
Schimburile culturale și relocările au devenit tot mai comune știind că străbaterea unei distanțe mari nu mai reprezintă un drum unidirecțional. Astfel s-a început o înceată dar sigură omogenizare a populației.
Transportul mărfurilor a devenit ușor și rapid, facilitând comerțul și crescând varietatea produselor.

Lumea s-a micșorat și mai mult cu introducerea călătoriei aeriene, acum devenind posibil să se ajungă oriunde pe glob în câteva ore, creându-se o nouă schimbare de percepție în privința distanțelor.

Pe de altă parte, dacă ne concentrăm asupra interacțiunii cu uneltele pe care le dezvoltăm, citatul capătă un nou înțeles.
Unelte rudimentare ca ciocanul sau sulița ne extind capabilitățile umane într-un mod natural, știind instant cum se folosesc de îndată ce au fost luate in mană, devenind o extensie a acesteia.
Dar trecând la unelte mai complexe, lucrurile devin mai puțin intuitive. O bicicletă, deși concepută ca o extensie a piciorului, va necesita acumularea unor abilități noi pentru a deveni utilă. Automobilul necesită învățarea unei multitudini de comenzi nenaturale sau legate de acțiunea pe care încearcă să o eficientizeze pentru a augmenta o abilitate naturală de bază a omului. În acest sens, noi ca utilizatori trebuie să parcurgem o parte din drum pentru a ne putea îmbunătății calitatea vieții prin adoptarea unor unelte noi, trebuie să ne adaptăm la propriile noastre unelte.

Astăzi computerele sunt pretutindeni, și indiferent de ocupația pe care o avem, interacționăm cu ele în fiecare zi. Nu trec mai mult de câteva minute, în medie, din momentul în care ne trezim, până avem în față un ecran.
Computerele evoluează pe zi ce trece și se înrădăcinează tot mai mult în stilul nostru de viață, dar din cauza numărul tot mai mare de interacțiuni necesare cu acestea și a volumului de informații la care suntem constant expuși, sistemele de comunicare cu computerele aduc o limitare a eficienței și a potențialului uman.

Pentru a minimiza compromisurile necesare adoptării unor instrumente în societate și a facilita utilizarea lor, trebuie să le modelam în funcție de aptitudinile noastre naturale.
Acest lucru este susținut și de Daniel G. Siegel, arhitect de produse digitale și consultant în interacțiune om-calculator, într-o prezentare intitulată \textit{The lost medium}: 
\begin{quotation}
„Nu e neapărat o problemă tehnologică, cât o lipsă de perspectivă. Am putea să rezolvăm probleme una câte una, sau am putea să facem ceva mai mult, cum ar fi să folosim computerele pentru a ne augmenta cele mai umane capabilități.
Trebuie să ne gândim ce forma vrem sa-i dăm computerului pentru a îmbunătăți societatea, cultura.”
\end{quotation}
