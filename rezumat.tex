\begin{abstract}

În această lucrare se explorează câteva concepte apărute odată cu era digitală, a căror înțelegere și recunoaștere afectează felul în care ne raportam la tehnologie.

Încercăm să conturăm o distincție intre \textit{mediu}\footnote{\textit{Medium} (cuvânt de origine latină, care înseamnă și "loc public","domeniu comun") este forma la singular a cuvântului "media", formă fără echivalent în limba română, motiv pentru care am optat pentru "mediu" în loc de "medium".} și \textit{tehnologie}. Tehnologia reprezintă aparatura fizică, cu caracteristici măsurabile, care se îmbunătățesc cu fiecare iterație. Atunci când ea este introdusă în societate și începe să îndeplinească o utilitate, aceasta devine un mediu. Un nou mediu reprezintă mai mult decât un nou gadget cu care ne putem juca, acesta sculptează stilul de viață și percepțiile asupra normalității ale tuturor celor care-l adoptă.

Realitatea virtuală și cea augmentată au potențialul de a înlocui unele din aceste metode. Deși nu sunt concepte noi, au fost până acum cu câțiva pași în spatele tehnologiei și readuse în lumina reflectoarelor de proiecte ca Google Glass și Oculus Rift. La scurt timp după expunerea publică a ideii de realitate virtuală/augmentată, numeroase companii și-au manifestat interesul în această nouă formă de interacțiune cu mediul virtual. Astfel, dispozitive similare cum ar fi HTC Vive, Project Morpheus, Microsoft Lens au început să își facă apariția pe piața de larg.

În încercarea de a exploata acest nou mediu, am dezvoltat un software prototip de vizualizare a resurselor web în spațiul virtual folosind Oculus Rift, dispozitivul Leap Motion (pentru captura și transpunerea mâinilor în mediul virtual ca metodă input) și motorul grafic Unreal Engine (ca bază pentru dezvoltarea software-ului). 

Printre cazurile de utilizare ale aplicației se numără explorarea unei galerii de artă, vizualizarea unui album foto și redarea conținutului video.

Aplicația a fost dezvoltată folosind un hibrid între sistemul vizual de scripting oferit de Unreal Engine folosit în principal pentru crearea elementelor grafice și cod C++ folosit pentru accesarea și aducerea resurselor web necesare generării elementelor grafice.

Platforma XWiki a fost utilizată pentru stocarea informațiilor web într-un mod structurat, accesarea resurselor fiind facilitată de XWiki RESTful API.

% Insert blank page
\newpage
\thispagestyle{empty}
\mbox{}

\end{abstract}

